%%% --------------------------------------------------------
%%> \section{Директиви компілятора}
%%% --------------------------------------------------------

% !TeX program = lualatex
% !TeX encoding = utf8
% !BIB TS-program = biber




\documentclass[]{iptconf}


\usepackage{multirow}


%%% --------------------------------------------------------
%%> \section{Реєстраційна форма}
%%% --------------------------------------------------------





\regform{
    fullname = {Аватар Великович Тор},          % Повне ім'я доповідача (перший автор)
    birthday = {01.02.1990},                    % Дата народження доповідача
    position = {асистент},                      % Посада доповідача
    phone = {+380935555555},                    % Телефонний номер доповідача
    authoremail = {mr_thor@gmail.com},          % Email доповідача
    confsection = {Фізика енергетичних систем}, % Секція конференції,
    copynum = {0},                              % Замовлена число друкованого збірника
    needliving = {ні},                          % Потреба в житлі (Ні/Хостел/Готель/інше)
    needinvitanion = {ні},                      % Чи потрібне запрошення на конференцію?
}





%%% --------------------------------------------------------
%%> \section{Використані пакети}
%%% --------------------------------------------------------





\usepackage[most]{tcolorbox}
\usepackage{tabularray}
\UseTblrLibrary{booktabs}
\usepackage{mathtools}
\usepackage{dsfont}
\usepackage{mathrsfs}
\usepackage{wrapfig}
\usepackage{xurl}
\usepackage[version=4]{mhchem}
\usepackage{forest}
\usepackage{tikz}
\usepackage{pgfplots}
\usepackage{listings}
\usetikzlibrary{shadows,arrows.meta}





%%% --------------------------------------------------------
%%> \section{Файл бібліографії}
%%% --------------------------------------------------------





%% Змініть ім'я файлу бібліографі на ваш.
%% Краще, щоб його назва була така ж сама,
%% як у вашого .tex-файлу

\addbibresource{author.bib}





%%% --------------------------------------------------------
%%> \section{Команди користувача}
%%% --------------------------------------------------------





%%% --------------------------------------------------------
%%> \section{Заголовок статті}
%%% --------------------------------------------------------





\title{Інтеграція функцій для розпізнавання активності на основі акселерометра}





%%% --------------------------------------------------------
%%> \section{Автори}
%%% --------------------------------------------------------



\author[]{C. Berke Erdas}{1}
\author{Isul Atasoy}{1}
\author{Koray Acic}{1}
\author{Hasan Ogul}{1}





%%% --------------------------------------------------------
%%> \section{Установи}
%%% --------------------------------------------------------




%% Тут введіть установу в якій працює, або навчається другий автор
\affiliation{Department of Computer Engineering, Başkent University, Ankara, Turkey}{1}





%%% --------------------------------------------------------
%%> \section{УДК та PACS}
%%% --------------------------------------------------------





%\pacs{ }
\udc{}





%%% --------------------------------------------------------
%%> \section{Анотація до статті}
%%% --------------------------------------------------------

\abstract{
Розпізнавання активності --- це проблема прогнозування поточних дій людини за допомогою датчиків руху, що носяться на тілі. Проблема розпізнавання активності зазвичай підходять як до задачі керованої класифікації, де дискримінантна модель навчається на основі відомих зразків і новий запит присвоюється відомій мітці активності з використанням вивченої моделі. Складним питанням тут є те, як наповнити цей класифікатор з фіксованою кількістю ознак, коли реальним вхідним сигналом є необроблений сигнал різної довжини. У цьому дослідженні ми розглядаємо три можливі набори ознак, а саме: часову, частотну та вейвлет-статистику, а також їх комбінації для представлення Сигнал руху, отриманий з показань акселерометра, який носять на грудях за допомогою мобільного телефону. На додаток до систематичного порівняння цих наборів функцій, ми також надаємо комплексну оцінку деяких кроків попередньої обробки, таких як фільтрація та відбір ознак. Результати показують, що подача на класифікатор випадкового лісу ансамблевої вибірки найбільш релевантних ознак часової та частотних ознак, витягнутих з необроблених даних, може забезпечити найвищу точність у реальному наборі даних.

}





%%% --------------------------------------------------------
%%> \section{Ключові слова}
%%% --------------------------------------------------------





\keywords{Activity recognition; accelerometer analysis; feature selection}




\begin{document}
\PaperLanguage{ukrainian} %

\section{Вступ}

Останніми роками спостерігається значна кількість досліджень, які зосереджені на моніторингу та розпізнаванні патернів людської активності, зібраних за допомогою датчиків руху. Технології розпізнавання активності застосовуються в різних сферах розпізнавання активності, наприклад, у сфері охорони здоров'я, догляду за людьми похилого віку чи спортивних пристроїв для відстеження руху. Багато попередніх досліджень пропонували використовувати датчик акселерометра для здійснення процесу розпізнавання. Акселерометри були широко поширеними пристроями для вимірювання особистих щоденних дій, таких як ходьба, стояння і біг, завдяки їх мінімальним розмірам, низькому енергоспоживанню, вартості і можливості отримувати дані безпосередньо з руху. Попередні дослідження показали, що методології машинного навчання ефективні для класифікації різних видів діяльності на основі сенсорних даних~\cite{1}-\cite{9}. Вони часто працюють у два етапи. По-перше, відповідні ознаки обчислюються з сигналу акселерометра даних акселерометра. Потім використовується алгоритм класифікатора для визначення активності, що відповідає цим ознакам. Загальні ознаки включають статистику, витягнуту з часового аналізу сигналу, частотного аналізу та вейвлет-аналізу. аналізу, який також називають частотно-часовим аналізом.

Раві та ін. працювали з часовими ознаками і вибрали лише середнє значення, стандартне відхилення, енергію та кореляцію для класифікації сигналів акселерометра за допомогою таблиць рішень, дерев рішень (C4.5), $K$-найближчих сусідів, машин опорних векторів та наївних байєсівських класифікаторів~\cite{1}. Казале та ін. працювали з часовими характеристиками для кожного часового ряду і досліджували найкращі характеристики для класифікації фізичної активності~\cite{2}. Їхніми ознаками були середнє квадратичне значення та середнє значення мінімальної та максимальної суми. Для класифікації вони використовували алгоритм випадкового лісу. У дослідженні Пріса та ін. порівнювали дискримінаційну здатність часово-частотних ознак на основі фізичних навантажень~\cite{3}. Вони повідомили, що використання часових ознак може забезпечити досить хорошу точність. Ван та ін. використовували ознаки на основі ансамблевого емпіричного розкладання мод (EEMD) для класифікації сигналів тривісного акселерометра для розпізнавання активності~\cite{6}.

У цьому дослідженні ми ставили за мету (1) порівняти індивідуальний внесок наборів ознак, витягнутих з часових, частотних і часово-частотних представлень сигналів, зібраних за допомогою акселерометра, що носиться на тілі, (2) порівняти ефективність різних класифікаторів машинного навчання з точки зору точності прогнозування, (3) оцінити внесок деяких етапів попередньої обробки, таких як фільтрація і вибір ознак, в ефективність розпізнавання активності, і, нарешті, (4) виділити найбільш репрезентативну підмножину ознак з об'єднаної множини ознак, витягнутих з усіх представлень доменів. Результати показують, що найкращої точності можна досягти за допомогою обраної підмножини ознак з часової та частотної областей, коли вони подаються в класифікатор Random Forest без будь-якої попередньої обробки.


\section{Materials and methods}

\subsection{General overview}

Задача розпізнавання активності розглядається як задача керованої класифікації, де послідовність показань акселерометра подається на класифікатор машинного навчання. Вхідні дані нормалізуються так, щоб їхнє середнє значення дорівнювало нулю, а стандартне відхилення --- одиниці. Ознаки витягуються з сегментованих частин нормалізованих даних, де сегмент позначає кількість послідовних зчитувань акселерометра. Використовуються сегменти фіксованої довжини, оскільки немає попередніх знань про межі активності. Припускаючи, що будь-яка активність може демонструвати принаймні один зі своїх циклів за 4 секунди, кожен сегмент побудовано так, щоб мати 208 відліків. Як і в попередніх роботах, між двома послідовними вибірками допускається перекриття на 50\% довжини. На етапі класифікації ми використовуємо декілька класифікаторів машинного навчання, а саме: Random Forest, $k$-Nearest Neighbor (kNN) та Support Vector Machine (SVM).


\subsection{Feature extraction}

\subsubsection{Time domain features}

Ми вилучаємо 17 часових характеристик з кожного вікна для кожної осі $x$, $y$ та $z$. Керуючись попередньою роботою~\cite{8}, окремі характеристики для кожної осі включають статистичні атрибути, такі як середнє значення, дисперсія, стандартне відхилення та огинаючі метрики, тобто медіану, максимальне та мінімальне значення діапазону, метрику кореня головного квадрата. Крім того, ми використовуємо площу сигналу, індекси мінімального та максимального значення, потужність, енергію, ентропію, асиметрію, ексцес, куртоси, міжквартильний розмах та середнє абсолютне відхилення сигналу. Щоб побачити перехресний вплив різних осей руху, ми також використовуємо перехресну кореляцію бінарних комбінацій $x$, $y$ і $z$.

\subsubsection{Frequency domain features}

Ми виділяємо шість характеристик частотної області з кожного вікна для кожної осі $x$, $y$ та $z$. По-перше, швидке перетворення Фур'є використовується для перетворення даних у частотну область з часової. Першою характеристикою в частотній області є потужність сигналу в смузі частот. Другою вибраною характеристикою частотної області є енергія. Енергія визначається як сума квадратів параметрів FFT, які називаються коефіцієнтами. Іншою характеристикою є амплітуда, яка означає міру нормованого значення коефіцієнтів FFT і полегшує розпізнавання відмінностей між видами діяльності~\cite{3}. Остаточні характеристики частотної області визначаються як середнє, максимальне і мінімальне значення сигналу. Характеристикою постійного струму є середнє значення прискорення сигналу~\cite{4}.

\subsubsection{Time-frequency (wavelet) domain features}

Ми вилучаємо дев'ять наборів ознак у часо-частотній області з кожного вікна для кожної осі $x$, $y$ та $z$. Спочатку використовується дискретне вейвлет-перетворення для перетворення даних у часо-частотну область з часової області. Як і в попередніх дослідженнях, було обрано найточніший набір з трьох ознак. Ознаки, що обираються, були запропоновані в дослідженні Тамури та ін.~\cite{5}. У цьому дослідженні ці ознаки описуються як вимірювання потужності сигналу. Потужність сигналу була отримана шляхом підсумовування квадратів детальних коефіцієнтів де на четвертому та п'ятому рівнях вейвлет-перетворення~\cite{3}. Цей набір використовує Daubechies 3 як вейвлет-матір і дає загалом 6 ознак. Другий набір --- це квадрати коефіцієнтів. Його отримують шляхом розкладання сигналу на п'ять рівнів за допомогою материнського вейвлет-перетворення Daubechies 2. Потім ми підсумовуємо квадрати деталізованих коефіцієнтів для кожного з п'яти рівнів. Таким чином, в кінцевому підсумку ми отримуємо 15 ознак3. Останній набір ознак обчислюється шляхом підсумовування абсолютних значень детальних коефіцієнтів для кожного рівня. Він знову використовує Daubechies 2 як вейвлет-матір і видає 15 ознак в кінці.

\subsection{Фільтрація даних}

Щоб уникнути дефектів, які можуть бути спричинені шумами, і мати можливість дослідити відмінності між фільтрованими та нефільтрованими класифікаціями даних, ми розглядаємо можливість використання цифрового фільтра. Відповідно до характеристик наших даних, ми визначаємо частоту відсікання як $1$~Гц та певну частоту дискретизації $f$. Фільтр високих та низьких частот застосовується до даних окремо, і ознаки знову обчислюються в трьох областях, про які ми згадували вище. Результати порівнюються з точністю класифікації з фільтрацією та без неї для $f=52$ та $f=208$.


\subsection{Класифікація}

Завданням алгоритму розпізнавання активності є класифікація вхідного сигналу до одного з заданих класів активності. Для експериментів з різними класифікаторами ми використовуємо випадковий ліс, $k$-найближчого сусіда та машину опорних векторів.

Метод випадкового лісу будує ряд множинних дерев рішень для навчання моделі, де дерево рішень --- це структура, схожа на блок-схему, в якій кожен внутрішній вузол представляє тест на ознаці, що представляє відповідну вибірку~\cite{10}. Кожна гілка представляє результат тесту, а кожен листовий вузол --- мітку класу, тобто остаточне рішення за всіма оцінками ознак. Кожне дерево рішень будується за випадково вибраними значеннями з вхідних даних. Якщо вихідний вектор ознак має m ознак, кожне дерево використовує випадкову вибірку з $n$ ознак, які вибираються з усіх ознак. Деревам рішень дозволяється рости до тих пір, поки їх потужність не досягне $n$. Після навчання ліс дозволяє пропускати через себе кожен тестовий рядок, щоб вивести прогноз. Запит класифікується шляхом голосування над побудованими деревами рішень.

$k$-Nearest Neighbor --- один з алгоритмів лінивого навчання на основі екземплярів. Алгоритм починається з перевірки міток класів k найближчих сусідів у навчальній вибірці. Вибірка запиту класифікується за голосами сусідів. Клас, який отримує максимальну кількість голосів, призначається як передбачуваний клас для вибірки запиту. У нашому експерименті для порівняння векторів ознак обрано евклідову відстань.

Методологія SVM-класифікації --- це двоетапний процес. По-перше, вхідні дані класифікатора високої розмірності нелінійно відображаються в інший простір ознак. По-друге, з цього простору ознак будується нова лінійна гіперплощина з максимальним відривом для розділення класів екземплярів. Алгоритм SVM використовує опорні вектори, тоді як інші подібні алгоритми, такі як нейронні мережі, повинні перевіряти всі можливості гіперплощин для побудови поверхні рішень. SVM відомий як менш схильний до надмірного припасування, ніж деякі інші алгоритми.

\subsection{Вибір функцій}

Відбір ознак --- це завдання створення скороченої і, можливо, більш інформативної підмножини всіх ознак з усієї вибірки даних. Доведено, що для отримання точних і надійних результатів класифікації це є критично важливою потребою в ряді застосувань\cite{11, 12}. Враховуючи різноманітність методів відбору ознак в літературі, ми використовуємо ансамблевий підхід, що базується на консенсусі кількох поширених методів відбору ознак (Рисунок 1). Для цього ми подаємо всі 110 ознак, витягнутих з часової області, частотної області та часово-частотної області, до п'яти моделей відбору~\cite{11}, а саме: відбір за критерієм $\chi$-квадрат, відбір на основі кореляції (CFS), відбір на основі рельєфу (ReliefF), відбір на основі інформаційного підсилення (InfoGain) та відбір на основі коефіцієнта підсилення (GainRatio) окремо, а потім отримуємо їхній консенсусний список з їхніх вихідних даних. Ця схема забезпечує більш надійний відбір ознак, хоча кожен окремий метод відбору може давати різні результати.


\begin{table}[ht]
\centering
\caption{Effect of filtering on classification performance.}
\begin{tabular}{|l|l|l|l|l|l|l|l|}
\hline
Features & Filtering applied & None & High-Pass (f=208) & Low-Pass (f=208) & High-Pass (f=52) & Low-Pass (f=52) \\
\hline
\multirow{2}{*}{Time Domain} & Frequency Domain & 87\% & 86\% & 82\% & 74\% & 84\% \\
\cline{2-7}
& Wavelet Domain & 84\% & 83\% & 80\% & 75\% & 81\% \\
\hline
\multirow{2}{*}{Time+Wavelet Domain} & Frequency Domain & 52\% & 51\% & 57\% & 52\% & 54\% \\
\cline{2-7}
& Frequency+Wavelet Domain & 86\% & 84\% & 81\% & 73\% & 84\% \\
\hline
\end{tabular}
\end{table}



\end{document}