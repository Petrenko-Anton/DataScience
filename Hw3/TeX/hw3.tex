%============================ Compiler Directives =======================%%
%%                                                                        %%
% !TeX program = pdflatex
% !TeX encoding = utf8
% !TeX spellcheck = uk_UA
%%                                                                        %%
%%============================== Клас документа ==========================%%
%%                                                                        %%
\documentclass[]{article}
\usepackage[fontsize = 12pt]{fontsize}
\usepackage{ifluatex}
%%                                                                        %%
%%========================== Мови, шрифти та кодування ===================%%
%%
\ifluatex                                                                        %%
	\usepackage{fontspec}
	\setsansfont{CMU Sans Serif}%{Arial}
	\setmainfont{CMU Serif}%{Times New Roman}
	\setmonofont{CMU Typewriter Text}%{Consolas}
	\defaultfontfeatures{Ligatures={TeX}}
	\usepackage[math-style=TeX]{unicode-math}
\else
	\usepackage[utf8]{inputenc}
	\usepackage[T2A,T1]{fontenc}
	\usepackage{amsmath}
	%\usepackage{pscyr}
	\usepackage{cmap}
\fi
\usepackage[english, russian, ukrainian]{babel}

    \usepackage[breakable]{tcolorbox}
    \usepackage{parskip} % Stop auto-indenting (to mimic markdown behaviour)


    % Basic figure setup, for now with no caption control since it's done
    % automatically by Pandoc (which extracts ![](path) syntax from Markdown).
    \usepackage{graphicx}
    % Maintain compatibility with old templates. Remove in nbconvert 6.0
    \let\Oldincludegraphics\includegraphics
    % Ensure that by default, figures have no caption (until we provide a
    % proper Figure object with a Caption API and a way to capture that
    % in the conversion process - todo).
    \usepackage{caption}
    \DeclareCaptionFormat{nocaption}{}
    \captionsetup{format=nocaption,aboveskip=0pt,belowskip=0pt}

    \usepackage{float}
    \floatplacement{figure}{H} % forces figures to be placed at the correct location
    \usepackage{xcolor} % Allow colors to be defined
    \usepackage{enumerate} % Needed for markdown enumerations to work
    \usepackage{geometry} % Used to adjust the document margins
    \usepackage{amsmath} % Equations
    \usepackage{amssymb} % Equations
    \usepackage{textcomp} % defines textquotesingle
    % Hack from http://tex.stackexchange.com/a/47451/13684:
    \AtBeginDocument{%
        \def\PYZsq{\textquotesingle}% Upright quotes in Pygmentized code
    }
    \usepackage{upquote} % Upright quotes for verbatim code
    \usepackage{eurosym} % defines \euro
    \usepackage{fancyvrb} % verbatim replacement that allows latex
    \usepackage{grffile} % extends the file name processing of package graphics
                         % to support a larger range
    \makeatletter % fix for old versions of grffile with XeLaTeX
    \@ifpackagelater{grffile}{2019/11/01}
    {
      % Do nothing on new versions
    }
    {
      \def\Gread@@xetex#1{%
        \IfFileExists{"\Gin@base".bb}%
        {\Gread@eps{\Gin@base.bb}}%
        {\Gread@@xetex@aux#1}%
      }
    }
    \makeatother
    \usepackage[Export]{adjustbox} % Used to constrain images to a maximum size
    \adjustboxset{max size={0.9\linewidth}{0.9\paperheight}}

    % The hyperref package gives us a pdf with properly built
    % internal navigation ('pdf bookmarks' for the table of contents,
    % internal cross-reference links, web links for URLs, etc.)
    \usepackage{hyperref}
    % The default LaTeX title has an obnoxious amount of whitespace. By default,
    % titling removes some of it. It also provides customization options.
    \usepackage{titling}
    \usepackage{longtable} % longtable support required by pandoc >1.10
    \usepackage{booktabs}  % table support for pandoc > 1.12.2
    \usepackage{array}     % table support for pandoc >= 2.11.3
    \usepackage{calc}      % table minipage width calculation for pandoc >= 2.11.1
    \usepackage[inline]{enumitem} % IRkernel/repr support (it uses the enumerate* environment)
    \usepackage[normalem]{ulem} % ulem is needed to support strikethroughs (\sout)
                                % normalem makes italics be italics, not underlines
    \usepackage{mathrsfs}



    % Colors for the hyperref package
    \definecolor{urlcolor}{rgb}{0,.145,.698}
    \definecolor{linkcolor}{rgb}{.71,0.21,0.01}
    \definecolor{citecolor}{rgb}{.12,.54,.11}

    % ANSI colors
    \definecolor{ansi-black}{HTML}{3E424D}
    \definecolor{ansi-black-intense}{HTML}{282C36}
    \definecolor{ansi-red}{HTML}{E75C58}
    \definecolor{ansi-red-intense}{HTML}{B22B31}
    \definecolor{ansi-green}{HTML}{00A250}
    \definecolor{ansi-green-intense}{HTML}{007427}
    \definecolor{ansi-yellow}{HTML}{DDB62B}
    \definecolor{ansi-yellow-intense}{HTML}{B27D12}
    \definecolor{ansi-blue}{HTML}{208FFB}
    \definecolor{ansi-blue-intense}{HTML}{0065CA}
    \definecolor{ansi-magenta}{HTML}{D160C4}
    \definecolor{ansi-magenta-intense}{HTML}{A03196}
    \definecolor{ansi-cyan}{HTML}{60C6C8}
    \definecolor{ansi-cyan-intense}{HTML}{258F8F}
    \definecolor{ansi-white}{HTML}{C5C1B4}
    \definecolor{ansi-white-intense}{HTML}{A1A6B2}
    \definecolor{ansi-default-inverse-fg}{HTML}{FFFFFF}
    \definecolor{ansi-default-inverse-bg}{HTML}{000000}

    % common color for the border for error outputs.
    \definecolor{outerrorbackground}{HTML}{FFDFDF}

    % commands and environments needed by pandoc snippets
    % extracted from the output of `pandoc -s`
    \providecommand{\tightlist}{%
      \setlength{\itemsep}{0pt}\setlength{\parskip}{0pt}}
    \DefineVerbatimEnvironment{Highlighting}{Verbatim}{commandchars=\\\{\}}
    % Add ',fontsize=\small' for more characters per line
    \newenvironment{Shaded}{}{}
    \newcommand{\KeywordTok}[1]{\textcolor[rgb]{0.00,0.44,0.13}{\textbf{{#1}}}}
    \newcommand{\DataTypeTok}[1]{\textcolor[rgb]{0.56,0.13,0.00}{{#1}}}
    \newcommand{\DecValTok}[1]{\textcolor[rgb]{0.25,0.63,0.44}{{#1}}}
    \newcommand{\BaseNTok}[1]{\textcolor[rgb]{0.25,0.63,0.44}{{#1}}}
    \newcommand{\FloatTok}[1]{\textcolor[rgb]{0.25,0.63,0.44}{{#1}}}
    \newcommand{\CharTok}[1]{\textcolor[rgb]{0.25,0.44,0.63}{{#1}}}
    \newcommand{\StringTok}[1]{\textcolor[rgb]{0.25,0.44,0.63}{{#1}}}
    \newcommand{\CommentTok}[1]{\textcolor[rgb]{0.38,0.63,0.69}{\textit{{#1}}}}
    \newcommand{\OtherTok}[1]{\textcolor[rgb]{0.00,0.44,0.13}{{#1}}}
    \newcommand{\AlertTok}[1]{\textcolor[rgb]{1.00,0.00,0.00}{\textbf{{#1}}}}
    \newcommand{\FunctionTok}[1]{\textcolor[rgb]{0.02,0.16,0.49}{{#1}}}
    \newcommand{\RegionMarkerTok}[1]{{#1}}
    \newcommand{\ErrorTok}[1]{\textcolor[rgb]{1.00,0.00,0.00}{\textbf{{#1}}}}
    \newcommand{\NormalTok}[1]{{#1}}

    % Additional commands for more recent versions of Pandoc
    \newcommand{\ConstantTok}[1]{\textcolor[rgb]{0.53,0.00,0.00}{{#1}}}
    \newcommand{\SpecialCharTok}[1]{\textcolor[rgb]{0.25,0.44,0.63}{{#1}}}
    \newcommand{\VerbatimStringTok}[1]{\textcolor[rgb]{0.25,0.44,0.63}{{#1}}}
    \newcommand{\SpecialStringTok}[1]{\textcolor[rgb]{0.73,0.40,0.53}{{#1}}}
    \newcommand{\ImportTok}[1]{{#1}}
    \newcommand{\DocumentationTok}[1]{\textcolor[rgb]{0.73,0.13,0.13}{\textit{{#1}}}}
    \newcommand{\AnnotationTok}[1]{\textcolor[rgb]{0.38,0.63,0.69}{\textbf{\textit{{#1}}}}}
    \newcommand{\CommentVarTok}[1]{\textcolor[rgb]{0.38,0.63,0.69}{\textbf{\textit{{#1}}}}}
    \newcommand{\VariableTok}[1]{\textcolor[rgb]{0.10,0.09,0.49}{{#1}}}
    \newcommand{\ControlFlowTok}[1]{\textcolor[rgb]{0.00,0.44,0.13}{\textbf{{#1}}}}
    \newcommand{\OperatorTok}[1]{\textcolor[rgb]{0.40,0.40,0.40}{{#1}}}
    \newcommand{\BuiltInTok}[1]{{#1}}
    \newcommand{\ExtensionTok}[1]{{#1}}
    \newcommand{\PreprocessorTok}[1]{\textcolor[rgb]{0.74,0.48,0.00}{{#1}}}
    \newcommand{\AttributeTok}[1]{\textcolor[rgb]{0.49,0.56,0.16}{{#1}}}
    \newcommand{\InformationTok}[1]{\textcolor[rgb]{0.38,0.63,0.69}{\textbf{\textit{{#1}}}}}
    \newcommand{\WarningTok}[1]{\textcolor[rgb]{0.38,0.63,0.69}{\textbf{\textit{{#1}}}}}


    % Define a nice break command that doesn't care if a line doesn't already
    % exist.
    \def\br{\hspace*{\fill} \\* }
    % Math Jax compatibility definitions
    \def\gt{>}
    \def\lt{<}
    \let\Oldtex\TeX
    \let\Oldlatex\LaTeX
    \renewcommand{\TeX}{\textrm{\Oldtex}}
    \renewcommand{\LaTeX}{\textrm{\Oldlatex}}
    % Document parameters
    % Document title
    \title{hw3}





% Pygments definitions
\makeatletter
\def\PY@reset{\let\PY@it=\relax \let\PY@bf=\relax%
    \let\PY@ul=\relax \let\PY@tc=\relax%
    \let\PY@bc=\relax \let\PY@ff=\relax}
\def\PY@tok#1{\csname PY@tok@#1\endcsname}
\def\PY@toks#1+{\ifx\relax#1\empty\else%
    \PY@tok{#1}\expandafter\PY@toks\fi}
\def\PY@do#1{\PY@bc{\PY@tc{\PY@ul{%
    \PY@it{\PY@bf{\PY@ff{#1}}}}}}}
\def\PY#1#2{\PY@reset\PY@toks#1+\relax+\PY@do{#2}}

\@namedef{PY@tok@w}{\def\PY@tc##1{\textcolor[rgb]{0.73,0.73,0.73}{##1}}}
\@namedef{PY@tok@c}{\let\PY@it=\textit\def\PY@tc##1{\textcolor[rgb]{0.24,0.48,0.48}{##1}}}
\@namedef{PY@tok@cp}{\def\PY@tc##1{\textcolor[rgb]{0.61,0.40,0.00}{##1}}}
\@namedef{PY@tok@k}{\let\PY@bf=\textbf\def\PY@tc##1{\textcolor[rgb]{0.00,0.50,0.00}{##1}}}
\@namedef{PY@tok@kp}{\def\PY@tc##1{\textcolor[rgb]{0.00,0.50,0.00}{##1}}}
\@namedef{PY@tok@kt}{\def\PY@tc##1{\textcolor[rgb]{0.69,0.00,0.25}{##1}}}
\@namedef{PY@tok@o}{\def\PY@tc##1{\textcolor[rgb]{0.40,0.40,0.40}{##1}}}
\@namedef{PY@tok@ow}{\let\PY@bf=\textbf\def\PY@tc##1{\textcolor[rgb]{0.67,0.13,1.00}{##1}}}
\@namedef{PY@tok@nb}{\def\PY@tc##1{\textcolor[rgb]{0.00,0.50,0.00}{##1}}}
\@namedef{PY@tok@nf}{\def\PY@tc##1{\textcolor[rgb]{0.00,0.00,1.00}{##1}}}
\@namedef{PY@tok@nc}{\let\PY@bf=\textbf\def\PY@tc##1{\textcolor[rgb]{0.00,0.00,1.00}{##1}}}
\@namedef{PY@tok@nn}{\let\PY@bf=\textbf\def\PY@tc##1{\textcolor[rgb]{0.00,0.00,1.00}{##1}}}
\@namedef{PY@tok@ne}{\let\PY@bf=\textbf\def\PY@tc##1{\textcolor[rgb]{0.80,0.25,0.22}{##1}}}
\@namedef{PY@tok@nv}{\def\PY@tc##1{\textcolor[rgb]{0.10,0.09,0.49}{##1}}}
\@namedef{PY@tok@no}{\def\PY@tc##1{\textcolor[rgb]{0.53,0.00,0.00}{##1}}}
\@namedef{PY@tok@nl}{\def\PY@tc##1{\textcolor[rgb]{0.46,0.46,0.00}{##1}}}
\@namedef{PY@tok@ni}{\let\PY@bf=\textbf\def\PY@tc##1{\textcolor[rgb]{0.44,0.44,0.44}{##1}}}
\@namedef{PY@tok@na}{\def\PY@tc##1{\textcolor[rgb]{0.41,0.47,0.13}{##1}}}
\@namedef{PY@tok@nt}{\let\PY@bf=\textbf\def\PY@tc##1{\textcolor[rgb]{0.00,0.50,0.00}{##1}}}
\@namedef{PY@tok@nd}{\def\PY@tc##1{\textcolor[rgb]{0.67,0.13,1.00}{##1}}}
\@namedef{PY@tok@s}{\def\PY@tc##1{\textcolor[rgb]{0.73,0.13,0.13}{##1}}}
\@namedef{PY@tok@sd}{\let\PY@it=\textit\def\PY@tc##1{\textcolor[rgb]{0.73,0.13,0.13}{##1}}}
\@namedef{PY@tok@si}{\let\PY@bf=\textbf\def\PY@tc##1{\textcolor[rgb]{0.64,0.35,0.47}{##1}}}
\@namedef{PY@tok@se}{\let\PY@bf=\textbf\def\PY@tc##1{\textcolor[rgb]{0.67,0.36,0.12}{##1}}}
\@namedef{PY@tok@sr}{\def\PY@tc##1{\textcolor[rgb]{0.64,0.35,0.47}{##1}}}
\@namedef{PY@tok@ss}{\def\PY@tc##1{\textcolor[rgb]{0.10,0.09,0.49}{##1}}}
\@namedef{PY@tok@sx}{\def\PY@tc##1{\textcolor[rgb]{0.00,0.50,0.00}{##1}}}
\@namedef{PY@tok@m}{\def\PY@tc##1{\textcolor[rgb]{0.40,0.40,0.40}{##1}}}
\@namedef{PY@tok@gh}{\let\PY@bf=\textbf\def\PY@tc##1{\textcolor[rgb]{0.00,0.00,0.50}{##1}}}
\@namedef{PY@tok@gu}{\let\PY@bf=\textbf\def\PY@tc##1{\textcolor[rgb]{0.50,0.00,0.50}{##1}}}
\@namedef{PY@tok@gd}{\def\PY@tc##1{\textcolor[rgb]{0.63,0.00,0.00}{##1}}}
\@namedef{PY@tok@gi}{\def\PY@tc##1{\textcolor[rgb]{0.00,0.52,0.00}{##1}}}
\@namedef{PY@tok@gr}{\def\PY@tc##1{\textcolor[rgb]{0.89,0.00,0.00}{##1}}}
\@namedef{PY@tok@ge}{\let\PY@it=\textit}
\@namedef{PY@tok@gs}{\let\PY@bf=\textbf}
\@namedef{PY@tok@gp}{\let\PY@bf=\textbf\def\PY@tc##1{\textcolor[rgb]{0.00,0.00,0.50}{##1}}}
\@namedef{PY@tok@go}{\def\PY@tc##1{\textcolor[rgb]{0.44,0.44,0.44}{##1}}}
\@namedef{PY@tok@gt}{\def\PY@tc##1{\textcolor[rgb]{0.00,0.27,0.87}{##1}}}
\@namedef{PY@tok@err}{\def\PY@bc##1{{\setlength{\fboxsep}{\string -\fboxrule}\fcolorbox[rgb]{1.00,0.00,0.00}{1,1,1}{\strut ##1}}}}
\@namedef{PY@tok@kc}{\let\PY@bf=\textbf\def\PY@tc##1{\textcolor[rgb]{0.00,0.50,0.00}{##1}}}
\@namedef{PY@tok@kd}{\let\PY@bf=\textbf\def\PY@tc##1{\textcolor[rgb]{0.00,0.50,0.00}{##1}}}
\@namedef{PY@tok@kn}{\let\PY@bf=\textbf\def\PY@tc##1{\textcolor[rgb]{0.00,0.50,0.00}{##1}}}
\@namedef{PY@tok@kr}{\let\PY@bf=\textbf\def\PY@tc##1{\textcolor[rgb]{0.00,0.50,0.00}{##1}}}
\@namedef{PY@tok@bp}{\def\PY@tc##1{\textcolor[rgb]{0.00,0.50,0.00}{##1}}}
\@namedef{PY@tok@fm}{\def\PY@tc##1{\textcolor[rgb]{0.00,0.00,1.00}{##1}}}
\@namedef{PY@tok@vc}{\def\PY@tc##1{\textcolor[rgb]{0.10,0.09,0.49}{##1}}}
\@namedef{PY@tok@vg}{\def\PY@tc##1{\textcolor[rgb]{0.10,0.09,0.49}{##1}}}
\@namedef{PY@tok@vi}{\def\PY@tc##1{\textcolor[rgb]{0.10,0.09,0.49}{##1}}}
\@namedef{PY@tok@vm}{\def\PY@tc##1{\textcolor[rgb]{0.10,0.09,0.49}{##1}}}
\@namedef{PY@tok@sa}{\def\PY@tc##1{\textcolor[rgb]{0.73,0.13,0.13}{##1}}}
\@namedef{PY@tok@sb}{\def\PY@tc##1{\textcolor[rgb]{0.73,0.13,0.13}{##1}}}
\@namedef{PY@tok@sc}{\def\PY@tc##1{\textcolor[rgb]{0.73,0.13,0.13}{##1}}}
\@namedef{PY@tok@dl}{\def\PY@tc##1{\textcolor[rgb]{0.73,0.13,0.13}{##1}}}
\@namedef{PY@tok@s2}{\def\PY@tc##1{\textcolor[rgb]{0.73,0.13,0.13}{##1}}}
\@namedef{PY@tok@sh}{\def\PY@tc##1{\textcolor[rgb]{0.73,0.13,0.13}{##1}}}
\@namedef{PY@tok@s1}{\def\PY@tc##1{\textcolor[rgb]{0.73,0.13,0.13}{##1}}}
\@namedef{PY@tok@mb}{\def\PY@tc##1{\textcolor[rgb]{0.40,0.40,0.40}{##1}}}
\@namedef{PY@tok@mf}{\def\PY@tc##1{\textcolor[rgb]{0.40,0.40,0.40}{##1}}}
\@namedef{PY@tok@mh}{\def\PY@tc##1{\textcolor[rgb]{0.40,0.40,0.40}{##1}}}
\@namedef{PY@tok@mi}{\def\PY@tc##1{\textcolor[rgb]{0.40,0.40,0.40}{##1}}}
\@namedef{PY@tok@il}{\def\PY@tc##1{\textcolor[rgb]{0.40,0.40,0.40}{##1}}}
\@namedef{PY@tok@mo}{\def\PY@tc##1{\textcolor[rgb]{0.40,0.40,0.40}{##1}}}
\@namedef{PY@tok@ch}{\let\PY@it=\textit\def\PY@tc##1{\textcolor[rgb]{0.24,0.48,0.48}{##1}}}
\@namedef{PY@tok@cm}{\let\PY@it=\textit\def\PY@tc##1{\textcolor[rgb]{0.24,0.48,0.48}{##1}}}
\@namedef{PY@tok@cpf}{\let\PY@it=\textit\def\PY@tc##1{\textcolor[rgb]{0.24,0.48,0.48}{##1}}}
\@namedef{PY@tok@c1}{\let\PY@it=\textit\def\PY@tc##1{\textcolor[rgb]{0.24,0.48,0.48}{##1}}}
\@namedef{PY@tok@cs}{\let\PY@it=\textit\def\PY@tc##1{\textcolor[rgb]{0.24,0.48,0.48}{##1}}}

\def\PYZbs{\char`\\}
\def\PYZus{\char`\_}
\def\PYZob{\char`\{}
\def\PYZcb{\char`\}}
\def\PYZca{\char`\^}
\def\PYZam{\char`\&}
\def\PYZlt{\char`\<}
\def\PYZgt{\char`\>}
\def\PYZsh{\char`\#}
\def\PYZpc{\char`\%}
\def\PYZdl{\char`\$}
\def\PYZhy{\char`\-}
\def\PYZsq{\char`\'}
\def\PYZdq{\char`\"}
\def\PYZti{\char`\~}
% for compatibility with earlier versions
\def\PYZat{@}
\def\PYZlb{[}
\def\PYZrb{]}
\makeatother


    % For linebreaks inside Verbatim environment from package fancyvrb.
    \makeatletter
        \newbox\Wrappedcontinuationbox
        \newbox\Wrappedvisiblespacebox
        \newcommand*\Wrappedvisiblespace {\textcolor{red}{\textvisiblespace}}
        \newcommand*\Wrappedcontinuationsymbol {\textcolor{red}{\llap{\tiny$\m@th\hookrightarrow$}}}
        \newcommand*\Wrappedcontinuationindent {3ex }
        \newcommand*\Wrappedafterbreak {\kern\Wrappedcontinuationindent\copy\Wrappedcontinuationbox}
        % Take advantage of the already applied Pygments mark-up to insert
        % potential linebreaks for TeX processing.
        %        {, <, #, %, $, ' and ": go to next line.
        %        _, }, ^, &, >, - and ~: stay at end of broken line.
        % Use of \textquotesingle for straight quote.
        \newcommand*\Wrappedbreaksatspecials {%
            \def\PYGZus{\discretionary{\char`\_}{\Wrappedafterbreak}{\char`\_}}%
            \def\PYGZob{\discretionary{}{\Wrappedafterbreak\char`\{}{\char`\{}}%
            \def\PYGZcb{\discretionary{\char`\}}{\Wrappedafterbreak}{\char`\}}}%
            \def\PYGZca{\discretionary{\char`\^}{\Wrappedafterbreak}{\char`\^}}%
            \def\PYGZam{\discretionary{\char`\&}{\Wrappedafterbreak}{\char`\&}}%
            \def\PYGZlt{\discretionary{}{\Wrappedafterbreak\char`\<}{\char`\<}}%
            \def\PYGZgt{\discretionary{\char`\>}{\Wrappedafterbreak}{\char`\>}}%
            \def\PYGZsh{\discretionary{}{\Wrappedafterbreak\char`\#}{\char`\#}}%
            \def\PYGZpc{\discretionary{}{\Wrappedafterbreak\char`\%}{\char`\%}}%
            \def\PYGZdl{\discretionary{}{\Wrappedafterbreak\char`\$}{\char`\$}}%
            \def\PYGZhy{\discretionary{\char`\-}{\Wrappedafterbreak}{\char`\-}}%
            \def\PYGZsq{\discretionary{}{\Wrappedafterbreak\textquotesingle}{\textquotesingle}}%
            \def\PYGZdq{\discretionary{}{\Wrappedafterbreak\char`\"}{\char`\"}}%
            \def\PYGZti{\discretionary{\char`\~}{\Wrappedafterbreak}{\char`\~}}%
        }
        % Some characters . , ; ? ! / are not pygmentized.
        % This macro makes them "active" and they will insert potential linebreaks
        \newcommand*\Wrappedbreaksatpunct {%
            \lccode`\~`\.\lowercase{\def~}{\discretionary{\hbox{\char`\.}}{\Wrappedafterbreak}{\hbox{\char`\.}}}%
            \lccode`\~`\,\lowercase{\def~}{\discretionary{\hbox{\char`\,}}{\Wrappedafterbreak}{\hbox{\char`\,}}}%
            \lccode`\~`\;\lowercase{\def~}{\discretionary{\hbox{\char`\;}}{\Wrappedafterbreak}{\hbox{\char`\;}}}%
            \lccode`\~`\:\lowercase{\def~}{\discretionary{\hbox{\char`\:}}{\Wrappedafterbreak}{\hbox{\char`\:}}}%
            \lccode`\~`\?\lowercase{\def~}{\discretionary{\hbox{\char`\?}}{\Wrappedafterbreak}{\hbox{\char`\?}}}%
            \lccode`\~`\!\lowercase{\def~}{\discretionary{\hbox{\char`\!}}{\Wrappedafterbreak}{\hbox{\char`\!}}}%
            \lccode`\~`\/\lowercase{\def~}{\discretionary{\hbox{\char`\/}}{\Wrappedafterbreak}{\hbox{\char`\/}}}%
            \catcode`\.\active
            \catcode`\,\active
            \catcode`\;\active
            \catcode`\:\active
            \catcode`\?\active
            \catcode`\!\active
            \catcode`\/\active
            \lccode`\~`\~
        }
    \makeatother

    \let\OriginalVerbatim=\Verbatim
    \makeatletter
    \renewcommand{\Verbatim}[1][1]{%
        %\parskip\z@skip
        \sbox\Wrappedcontinuationbox {\Wrappedcontinuationsymbol}%
        \sbox\Wrappedvisiblespacebox {\FV@SetupFont\Wrappedvisiblespace}%
        \def\FancyVerbFormatLine ##1{\hsize\linewidth
            \vtop{\raggedright\hyphenpenalty\z@\exhyphenpenalty\z@
                \doublehyphendemerits\z@\finalhyphendemerits\z@
                \strut ##1\strut}%
        }%
        % If the linebreak is at a space, the latter will be displayed as visible
        % space at end of first line, and a continuation symbol starts next line.
        % Stretch/shrink are however usually zero for typewriter font.
        \def\FV@Space {%
            \nobreak\hskip\z@ plus\fontdimen3\font minus\fontdimen4\font
            \discretionary{\copy\Wrappedvisiblespacebox}{\Wrappedafterbreak}
            {\kern\fontdimen2\font}%
        }%

        % Allow breaks at special characters using \PYG... macros.
        \Wrappedbreaksatspecials
        % Breaks at punctuation characters . , ; ? ! and / need catcode=\active
        \OriginalVerbatim[#1,codes*=\Wrappedbreaksatpunct]%
    }
    \makeatother

    % Exact colors from NB
    \definecolor{incolor}{HTML}{303F9F}
    \definecolor{outcolor}{HTML}{D84315}
    \definecolor{cellborder}{HTML}{CFCFCF}
    \definecolor{cellbackground}{HTML}{F7F7F7}

    % prompt
    \makeatletter
    \newcommand{\boxspacing}{\kern\kvtcb@left@rule\kern\kvtcb@boxsep}
    \makeatother
    \newcommand{\prompt}[4]{
        {\ttfamily\llap{{\color{#2}[#3]:\hspace{3pt}#4}}\vspace{-\baselineskip}}
    }



    % Prevent overflowing lines due to hard-to-break entities
    \sloppy
    % Setup hyperref package
    \hypersetup{
      breaklinks=true,  % so long urls are correctly broken across lines
      colorlinks=true,
      urlcolor=urlcolor,
      linkcolor=linkcolor,
      citecolor=citecolor,
      }
    % Slightly bigger margins than the latex defaults

    \geometry{verbose,tmargin=1in,bmargin=1in,lmargin=1in,rmargin=1in}



\begin{document}

\maketitle

\begin{tcolorbox}[breakable, size=fbox, boxrule=1pt, pad at break*=1mm,colback=cellbackground, colframe=cellborder]
\prompt{In}{incolor}{1}{\boxspacing}
\begin{Verbatim}[commandchars=\\\{\}]
\PY{k+kn}{import} \PY{n+nn}{numpy} \PY{k}{as} \PY{n+nn}{np}
\PY{k+kn}{import} \PY{n+nn}{pandas} \PY{k}{as} \PY{n+nn}{pd}
\PY{k+kn}{import} \PY{n+nn}{matplotlib}\PY{n+nn}{.}\PY{n+nn}{pyplot} \PY{k}{as} \PY{n+nn}{plt}
\PY{k+kn}{from} \PY{n+nn}{sklearn}\PY{n+nn}{.}\PY{n+nn}{linear\PYZus{}model} \PY{k+kn}{import} \PY{n}{LinearRegression}
\end{Verbatim}
\end{tcolorbox}

Якщо у вас є набір даних з \(m\) вибірок, кожна з яких називається
\(x^{(i)}\) (\(n\)-вимірний вектор), і вектор результатів \$ Y \$
(\(m\)-вимірний вектор), можна побудувати наступні матриці:

\begin{enumerate}
	\def\labelenumi{\arabic{enumi}.}
	\tightlist
	\item
	      Матриця ознак
\end{enumerate}

\begin{equation*}
\mathbf{X} =
	\begin{pmatrix}
		\vec{x}^{(1)} \\
		\vec{x}^{(2)} \\
		\vdots        \\
		\vec{x}^{(m)} \\
	\end{pmatrix}
	=
	\begin{pmatrix}
		1      & x_1^{(1)} & x_2^{(1)} & \ldots & x_n^{(1)} \\
		1      & x_1^{(2)} & x_2^{(2)} & \ldots & x_n^{(2)} \\
		\vdots & \vdots    & \vdots    & \ddots & \vdots    \\
		1      & x_1^{(m)} & x_2^{(m)} & \ldots & x_n^{(m)} \\
	\end{pmatrix}
\end{equation*}


\begin{enumerate}
	\def\labelenumi{\arabic{enumi}.}
	\setcounter{enumi}{1}
	\tightlist
	\item
	      Вектор результатів
\end{enumerate}

\[
	\vec{Y} =
	\begin{pmatrix}
		\vec{y}_1 \\
		\vec{y}_2 \\
		\vdots    \\
		\vec{y}_m \\
	\end{pmatrix}
\]

\begin{enumerate}
	\def\labelenumi{\arabic{enumi}.}
	\setcounter{enumi}{2}
	\tightlist
	\item
	      Вектор вагових коефіцієнтів
\end{enumerate}

\[
	\vec{w} =
	\begin{pmatrix}
		\vec{w}_0 \\
		\vec{w}_1 \\
		\vdots    \\
		\vec{w}_n \\
	\end{pmatrix}
\]

Наша задача --- проаналізувати, як залежить ціна на будинок \(h\)
залежно від площі \(x_1\), кількості ванних кімнат \(x_2\) та кількості
спалень \(x_2\).



\subsection{Функція гіпотези лінійної
	регресії}\label{ux444ux443ux43dux43aux446ux456ux44f-ux433ux456ux43fux43eux442ux435ux437ux438-ux43bux456ux43dux456ux439ux43dux43eux457-ux440ux435ux433ux440ux435ux441ux456ux457}

Функція має вигляд: \[ \vec{h}(\vec{w}, X) = X \vec{w}, \] де $ \vec{w}
$ --- вектор вагових коефіцієнтів, \$ X \$ --- вектор-стовпець векторів
ознак (матриця ознак).

\begin{tcolorbox}[breakable, size=fbox, boxrule=1pt, pad at break*=1mm,colback=cellbackground, colframe=cellborder]
\prompt{In}{incolor}{2}{\boxspacing}
\begin{Verbatim}[commandchars=\\\{\}]
\PY{k}{def} \PY{n+nf}{h}\PY{p}{(}\PY{n}{W}\PY{p}{,} \PY{n}{X}\PY{p}{)}\PY{p}{:}
\PY{+w}{    }\PY{l+s+sd}{\PYZdq{}\PYZdq{}\PYZdq{}}
\PY{l+s+sd}{    Calculate the hypothesis for linear regression.}

\PY{l+s+sd}{    Parameters:}
\PY{l+s+sd}{    W (numpy.ndarray): Weight vector (dimension: (n+1,)).}
\PY{l+s+sd}{    X (numpy.ndarray): Feature matrix (dimension: (m, n+1)).}

\PY{l+s+sd}{    Returns:}
\PY{l+s+sd}{    hypothesis (numpy.ndarray): Hypothesis values (dimension: (m,)).}
\PY{l+s+sd}{    \PYZdq{}\PYZdq{}\PYZdq{}}
\PY{k}{return} \PY{n}{X} \PY{o}{@} \PY{n}{W}
\end{Verbatim}
\end{tcolorbox}

\subsection{Функція
	втрат}\label{ux444ux443ux43dux43aux446ux456ux44f-ux432ux442ux440ux430ux442}

Функція має вигляд:

\[ J(\vec{w}) = \frac1{2m} \left( \vec{h}(\vec{w}, \mathbf{X}) - \vec{Y} \right)^2. \]

Функція втрати (loss function) є однією з ключових компонентів в задачах
машинного навчання і глибокого навчання, і вона виконує декілька
важливих функцій:

\begin{enumerate}
	\def\labelenumi{\arabic{enumi}.}
	\item
	      \textbf{Вимірювання помилки моделі}: Функція втрати визначає,
	      наскільки добре модель попереджує або класифікує дані в порівнянні зі
	      справжніми (очікуваними) значеннями. Вона обчислює різницю між
	      прогнозованими і справжніми результатами. Ця різниця, яку часто
	      називають ``помилкою'' або ``втратою'', вказує на те, наскільки
	      великою є розрозненість між моделлю та даними.
	\item
	      \textbf{Оптимізація моделі}: Однією з центральних задач в машинному
	      навчанні є оптимізація параметрів моделі так, щоб функція втрати була
	      мінімізована. Іншими словами, ми намагаємося знайти такі значення
	      параметрів моделі, які роблять прогнози якомога ближчими до справжніх
	      даних. Це досягається шляхом зменшення значення функції втрати.
	\item
	      \textbf{Оцінка якості моделі}: Функція втрати дозволяє оцінювати
	      якість моделі. Чим менше значення функції втрати, тим краще модель
	      вирішує задачу. Виміряння втрати на навчальних та тестових даних
	      допомагає визначити, наскільки добре модель узагальнює свої знання на
	      нових даних.
\end{enumerate}

\begin{tcolorbox}[breakable, size=fbox, boxrule=1pt, pad at break*=1mm,colback=cellbackground, colframe=cellborder]
\prompt{In}{incolor}{3}{\boxspacing}
\begin{Verbatim}[commandchars=\\\{\}]
\PY{k}{def} \PY{n+nf}{J}\PY{p}{(}\PY{n}{W}\PY{p}{,} \PY{n}{X}\PY{p}{,} \PY{n}{Y}\PY{p}{)}\PY{p}{:}
\PY{+w}{    }\PY{l+s+sd}{\PYZdq{}\PYZdq{}\PYZdq{}}
\PY{l+s+sd}{    Calculate the mean squared error (MSE) for linear regression.}

\PY{l+s+sd}{    Parameters:}
\PY{l+s+sd}{    W (numpy.ndarray): Weight vector (dimension: (n+1,)).}
\PY{l+s+sd}{    X (numpy.ndarray): Feature matrix (dimension: (m, n+1)).}
\PY{l+s+sd}{    Y (numpy.ndarray): Target vector (dimension: (m,)).}

\PY{l+s+sd}{    Returns:}
\PY{l+s+sd}{    mse (float): Mean squared error.}
\PY{l+s+sd}{    \PYZdq{}\PYZdq{}\PYZdq{}}
\PY{n}{m} \PY{o}{=} \PY{n+nb}{len}\PY{p}{(}\PY{n}{Y}\PY{p}{)}  \PY{c+c1}{\PYZsh{} Кількість навчальних прикладів}
\PY{n}{error} \PY{o}{=} \PY{n}{h}\PY{p}{(}\PY{n}{W}\PY{p}{,} \PY{n}{X}\PY{p}{)} \PY{o}{\PYZhy{}} \PY{n}{Y}
\PY{k}{return} \PY{l+m+mi}{1} \PY{o}{/} \PY{p}{(} \PY{l+m+mi}{2} \PY{o}{*} \PY{n}{m} \PY{p}{)} \PY{o}{*} \PY{n}{error} \PY{o}{@} \PY{n}{error}
\end{Verbatim}
\end{tcolorbox}

\subsection{Градієнт функції
	втрат}\label{ux433ux440ux430ux434ux456ux454ux43dux442-ux444ux443ux43dux43aux446ux456ux457-ux432ux442ux440ux430ux442}

Вектор-градієнт функції втрат має вигляд:

\[ \vec{\nabla} J = \frac1{m} \mathbf{X}^{\mathrm{T}} \cdot (\mathrm{X}\vec{w} - \vec{Y} )  = \frac1{m} \mathbf{X}^{\mathrm{T}} \cdot (\vec{h} - \vec{Y} ). \]

\textbf{Градієнт функції втрат} (gradient of the loss function) - це
вектор, який показує напрямок та швидкість найшвидшого зростання функції
втрат в околицях поточного значення параметрів моделі. Іншими словаим,
це вектор, який показує, як зміниться значення функції втрат при дуже
невеликих змінах параметрів моделі.

Основні аспекти градієнту функції втрат:

\begin{itemize}
	\item
	      \textbf{Напрямок}: Градієнт вказує напрямок найшвидшого зростання
	      функції втрат. Якщо ви рухаєтесь в напрямку градієнту, то значення
	      функції втрат буде зростати найшвидше.
	\item
	      \textbf{Величина}: Модуль градієнту (його довжина) показує, наскільки
	      швидко зростає функція втрат в цьому напрямку. Більший градієнт вказує
	      на більш значущі зміни в функції втрат при малих змінах параметрів
	      моделі.
\end{itemize}

Використання градієнту функції втрат дуже важливо в процесі оптимізації
моделі, так як він вказує на те, які кроки (зміни параметрів) потрібно
зробити для покращення моделі. У методах оптимізації, таких як
стохастичний градієнтний спуск, градієнт використовується для оновлення
параметрів моделі з метою зменшення функції втрат.

\begin{tcolorbox}[breakable, size=fbox, boxrule=1pt, pad at break*=1mm,colback=cellbackground, colframe=cellborder]
\prompt{In}{incolor}{4}{\boxspacing}
\begin{Verbatim}[commandchars=\\\{\}]
\PY{k}{def} \PY{n+nf}{nabla\PYZus{}J}\PY{p}{(}\PY{n}{W}\PY{p}{,} \PY{n}{X}\PY{p}{,} \PY{n}{Y}\PY{p}{)}\PY{p}{:}
\PY{+w}{    }\PY{l+s+sd}{\PYZdq{}\PYZdq{}\PYZdq{}}
\PY{l+s+sd}{    Computes the gradient of the loss function for linear regression.}

\PY{l+s+sd}{    Parameters:}
\PY{l+s+sd}{    W (numpy.ndarray): Vector of weights (dimensionality (n+1,)).}
\PY{l+s+sd}{    X (numpy.ndarray): Feature matrix (dimensionality (m, n+1)).}
\PY{l+s+sd}{    Y (numpy.ndarray): Target value vector (dimensionality (m,)).}

\PY{l+s+sd}{    Returns:}
\PY{l+s+sd}{    Gradient (numpy.ndarray): The gradient of the loss function (dimension (n+1,)).}
\PY{l+s+sd}{    \PYZdq{}\PYZdq{}\PYZdq{}}

\PY{n}{m} \PY{o}{=} \PY{n+nb}{len}\PY{p}{(}\PY{n}{Y}\PY{p}{)}  \PY{c+c1}{\PYZsh{} Кількість навчальних прикладів}
\PY{n}{error}  \PY{o}{=} \PY{n}{X}\PY{o}{.}\PY{n}{T} \PY{o}{@} \PY{p}{(}\PY{n}{h}\PY{p}{(}\PY{n}{W}\PY{p}{,} \PY{n}{X}\PY{p}{)} \PY{o}{\PYZhy{}} \PY{n}{Y}\PY{p}{)}
\PY{n}{gradient} \PY{o}{=} \PY{p}{(}\PY{l+m+mi}{1} \PY{o}{/} \PY{n}{m}\PY{p}{)} \PY{o}{*} \PY{n}{error}
\PY{k}{return} \PY{n}{gradient}
\end{Verbatim}
\end{tcolorbox}

\subsection{Функція градієнтного
	спуску}\label{ux444ux443ux43dux43aux446ux456ux44f-ux433ux440ux430ux434ux456ux454ux43dux442ux43dux43eux433ux43e-ux441ux43fux443ux441ux43aux443}

Формула для обчислення вагових коефіцієнтів в результаті одного кроку
градієнтного спуску (одна ітерація) має вигляд:

\[ \vec{w} = \vec{w}_{\text{prev}} - \alpha \vec{\nabla} J \]

\begin{tcolorbox}[breakable, size=fbox, boxrule=1pt, pad at break*=1mm,colback=cellbackground, colframe=cellborder]
\prompt{In}{incolor}{5}{\boxspacing}
\begin{Verbatim}[commandchars=\\\{\}]
\PY{k}{def} \PY{n+nf}{gradient\PYZus{}descent}\PY{p}{(}\PY{n}{X}\PY{p}{,} \PY{n}{Y}\PY{p}{,}
\PY{n}{alpha}\PY{o}{=}\PY{l+m+mf}{0.001}\PY{p}{,}
\PY{n}{num\PYZus{}iterations}\PY{o}{=}\PY{l+m+mi}{1\PYZus{}000}\PY{p}{,}
\PY{n}{epsilon}\PY{o}{=}\PY{l+m+mf}{1e\PYZhy{}7}\PY{p}{)}\PY{p}{:}
\PY{+w}{    }\PY{l+s+sd}{\PYZdq{}\PYZdq{}\PYZdq{}}
\PY{l+s+sd}{    Perform gradient descent optimization for linear regression.}

\PY{l+s+sd}{    Parameters:}
\PY{l+s+sd}{    X (numpy.ndarray): Feature matrix (dimension: (m, n+1)).}
\PY{l+s+sd}{    Y (numpy.ndarray): Target vector (dimension: (m,)).}
\PY{l+s+sd}{    alpha (float, optional): Learning rate. Defaults to 0.001.}
\PY{l+s+sd}{    num\PYZus{}iterations (int, optional): Number of iterations. Defaults to 1000.}
\PY{l+s+sd}{    epsilon (float, optional): Convergence threshold. Defaults to 1e\PYZhy{}7.}

\PY{l+s+sd}{    Returns:}
\PY{l+s+sd}{    W (numpy.ndarray): Optimized weight vector (dimension: (n+1,)).}
\PY{l+s+sd}{    history\PYZus{}J (list): List of loss values during optimization.}
\PY{l+s+sd}{    \PYZdq{}\PYZdq{}\PYZdq{}}

\PY{n}{n} \PY{o}{=} \PY{n}{X}\PY{o}{.}\PY{n}{shape}\PY{p}{[}\PY{l+m+mi}{1}\PY{p}{]}  \PY{c+c1}{\PYZsh{} Кількість ознак (у цьому випадку 3: area, bedrooms, bathrooms)}

\PY{c+c1}{\PYZsh{} Ініціалізуємо вагові коефіцієнти випадковими значеннями}
\PY{n}{W} \PY{o}{=} \PY{n}{np}\PY{o}{.}\PY{n}{random}\PY{o}{.}\PY{n}{randn}\PY{p}{(}\PY{n}{n}\PY{p}{)}

\PY{n}{J\PYZus{}0} \PY{o}{=} \PY{n}{J}\PY{p}{(}\PY{n}{W}\PY{p}{,} \PY{n}{X}\PY{p}{,} \PY{n}{Y}\PY{p}{)}

\PY{n}{history\PYZus{}J} \PY{o}{=} \PY{p}{[}\PY{n}{J\PYZus{}0}\PY{p}{]}

\PY{k}{for} \PY{n}{\PYZus{}} \PY{o+ow}{in} \PY{n+nb}{range}\PY{p}{(}\PY{n}{num\PYZus{}iterations}\PY{p}{)}\PY{p}{:}
\PY{c+c1}{\PYZsh{} Оновлюємо коефіцієнти}
\PY{n}{W} \PY{o}{\PYZhy{}}\PY{o}{=} \PY{n}{alpha} \PY{o}{*} \PY{n}{nabla\PYZus{}J}\PY{p}{(}\PY{n}{W}\PY{p}{,} \PY{n}{X}\PY{p}{,} \PY{n}{Y}\PY{p}{)}

\PY{n}{J\PYZus{}current} \PY{o}{=} \PY{n}{J}\PY{p}{(}\PY{n}{W}\PY{p}{,} \PY{n}{X}\PY{p}{,} \PY{n}{Y}\PY{p}{)}

\PY{n}{history\PYZus{}J}\PY{o}{.}\PY{n}{append}\PY{p}{(}\PY{n}{J\PYZus{}current}\PY{p}{)}

\PY{k}{if} \PY{n}{np}\PY{o}{.}\PY{n}{abs}\PY{p}{(}\PY{n}{J\PYZus{}current} \PY{o}{\PYZhy{}} \PY{n}{J\PYZus{}0}\PY{p}{)} \PY{o}{\PYZlt{}} \PY{n}{epsilon}\PY{p}{:}
\PY{k}{break}

\PY{n}{J\PYZus{}0} \PY{o}{=} \PY{n}{J\PYZus{}current}


\PY{k}{return} \PY{n}{W}\PY{p}{,} \PY{n}{history\PYZus{}J}
\end{Verbatim}
\end{tcolorbox}



\begin{tcolorbox}[breakable, size=fbox, boxrule=1pt, pad at break*=1mm,colback=cellbackground, colframe=cellborder]
\prompt{In}{incolor}{6}{\boxspacing}
\begin{Verbatim}[commandchars=\\\{\}]
\PY{n}{df} \PY{o}{=} \PY{n}{pd}\PY{o}{.}\PY{n}{read\PYZus{}csv}\PY{p}{(}\PY{l+s+s1}{\PYZsq{}}\PY{l+s+s1}{Housing.csv}\PY{l+s+s1}{\PYZsq{}}\PY{p}{)}
\PY{n}{X} \PY{o}{=} \PY{n}{df}\PY{p}{[}\PY{p}{[}\PY{l+s+s1}{\PYZsq{}}\PY{l+s+s1}{area}\PY{l+s+s1}{\PYZsq{}}\PY{p}{,} \PY{l+s+s1}{\PYZsq{}}\PY{l+s+s1}{bedrooms}\PY{l+s+s1}{\PYZsq{}}\PY{p}{,} \PY{l+s+s1}{\PYZsq{}}\PY{l+s+s1}{bathrooms}\PY{l+s+s1}{\PYZsq{}}\PY{p}{]}\PY{p}{]}\PY{o}{.}\PY{n}{to\PYZus{}numpy}\PY{p}{(}\PY{p}{)}
\PY{n}{Y} \PY{o}{=} \PY{n}{df}\PY{o}{.}\PY{n}{price}\PY{o}{.}\PY{n}{to\PYZus{}numpy}\PY{p}{(}\PY{p}{)}
\end{Verbatim}
\end{tcolorbox}

\subsection{Нормалізація
	даних}\label{ux43dux43eux440ux43cux430ux43bux456ux437ux430ux446ux456ux44f-ux434ux430ux43dux438ux445}

Щоб наша модель швидше навчалась, необхідно виконати нормалізацію даних,
оскільки \(x_1 = \text{area}\) сильно відрізняється за порядком від
\(x_2 = \text{bedrooms}\) та \(x_3 = \text{bathrooms}\).

Нормалізацію виконаємо за формулою:

\[ \mathrm{X}^{\text{norm}} = \frac{\mathrm{X} - \overline{\mathrm{X}}}{\sigma}, \]

де \(\overline{\mathrm{X}}\) - середнє (за стовбчиком), \(\sigma\) -
дисперсія (стандртне відхилення).

\begin{tcolorbox}[breakable, size=fbox, boxrule=1pt, pad at break*=1mm,colback=cellbackground, colframe=cellborder]
\prompt{In}{incolor}{7}{\boxspacing}
\begin{Verbatim}[commandchars=\\\{\}]
\PY{k}{def} \PY{n+nf}{normalize\PYZus{}features}\PY{p}{(}\PY{n}{X}\PY{p}{)}\PY{p}{:}
\PY{n}{mean} \PY{o}{=} \PY{n}{np}\PY{o}{.}\PY{n}{mean}\PY{p}{(}\PY{n}{X}\PY{p}{,} \PY{n}{axis}\PY{o}{=}\PY{l+m+mi}{0}\PY{p}{)}
\PY{n}{std} \PY{o}{=} \PY{n}{np}\PY{o}{.}\PY{n}{std}\PY{p}{(}\PY{n}{X}\PY{p}{,} \PY{n}{axis}\PY{o}{=}\PY{l+m+mi}{0}\PY{p}{)}

\PY{c+c1}{\PYZsh{} Перевіряємо, що стандартне відхилення не дорівнює нулю}
\PY{n}{std}\PY{p}{[}\PY{n}{std} \PY{o}{==} \PY{l+m+mi}{0}\PY{p}{]} \PY{o}{=} \PY{l+m+mi}{1}

\PY{n}{normalized\PYZus{}X} \PY{o}{=} \PY{p}{(}\PY{n}{X} \PY{o}{\PYZhy{}} \PY{n}{mean}\PY{p}{)} \PY{o}{/} \PY{n}{std}
\PY{k}{return} \PY{n}{normalized\PYZus{}X}\PY{p}{,} \PY{n}{mean}\PY{p}{,} \PY{n}{std}
\end{Verbatim}
\end{tcolorbox}

\begin{tcolorbox}[breakable, size=fbox, boxrule=1pt, pad at break*=1mm,colback=cellbackground, colframe=cellborder]
\prompt{In}{incolor}{8}{\boxspacing}
\begin{Verbatim}[commandchars=\\\{\}]
\PY{c+c1}{\PYZsh{} Нормалізуємо ознаки}
\PY{n}{X\PYZus{}n}\PY{p}{,} \PY{n}{mean}\PY{p}{,} \PY{n}{std} \PY{o}{=} \PY{n}{normalize\PYZus{}features}\PY{p}{(}\PY{n}{X}\PY{p}{)}

\PY{c+c1}{\PYZsh{} Додаємо стовпець з одиницями для вільного члена (bias)}
\PY{n}{X\PYZus{}n} \PY{o}{=} \PY{n}{np}\PY{o}{.}\PY{n}{column\PYZus{}stack}\PY{p}{(}\PY{p}{[}\PY{n}{np}\PY{o}{.}\PY{n}{ones}\PY{p}{(}\PY{n+nb}{len}\PY{p}{(}\PY{n}{X}\PY{p}{)}\PY{p}{)}\PY{p}{,} \PY{n}{X\PYZus{}n}\PY{p}{]}\PY{p}{)}
\end{Verbatim}
\end{tcolorbox}



\begin{tcolorbox}[breakable, size=fbox, boxrule=1pt, pad at break*=1mm,colback=cellbackground, colframe=cellborder]
\prompt{In}{incolor}{9}{\boxspacing}
\begin{Verbatim}[commandchars=\\\{\}]
\PY{c+c1}{\PYZsh{} Викликаємо функцію градієнтного спуску}
\PY{n}{learned\PYZus{}weights}\PY{p}{,} \PY{n}{history\PYZus{}J} \PY{o}{=} \PY{n}{gradient\PYZus{}descent}\PY{p}{(}\PY{n}{X\PYZus{}n}\PY{p}{,} \PY{n}{Y}\PY{p}{,} \PY{n}{num\PYZus{}iterations}\PY{o}{=}\PY{l+m+mi}{100\PYZus{}000}\PY{p}{)}

\PY{c+c1}{\PYZsh{} Відновлюємо ненормалізовані ваги}
\PY{n}{intercept} \PY{o}{=} \PY{n}{learned\PYZus{}weights}\PY{p}{[}\PY{l+m+mi}{0}\PY{p}{]}
\PY{n}{coefficients} \PY{o}{=} \PY{n}{learned\PYZus{}weights}\PY{p}{[}\PY{l+m+mi}{1}\PY{p}{:}\PY{p}{]} \PY{o}{/} \PY{n}{std}
\end{Verbatim}
\end{tcolorbox}

\subsubsection{Вагові коефіцієнти після градієнтного
	спуску}\label{ux432ux430ux433ux43eux432ux456-ux43aux43eux435ux444ux456ux446ux456ux454ux43dux442ux438-ux43fux456ux441ux43bux44f-ux433ux440ux430ux434ux456ux454ux43dux442ux43dux43eux433ux43e-ux441ux43fux443ux441ux43aux443}

\begin{tcolorbox}[breakable, size=fbox, boxrule=1pt, pad at break*=1mm,colback=cellbackground, colframe=cellborder]
\prompt{In}{incolor}{10}{\boxspacing}
\begin{Verbatim}[commandchars=\\\{\}]
\PY{n+nb}{print}\PY{p}{(}\PY{l+s+sa}{f}\PY{l+s+s2}{\PYZdq{}}\PY{l+s+s2}{Вільний член (intercept): }\PY{l+s+si}{\PYZob{}}\PY{n}{intercept}\PY{l+s+si}{\PYZcb{}}\PY{l+s+s2}{\PYZdq{}}\PY{p}{)}
\PY{n+nb}{print}\PY{p}{(}\PY{l+s+sa}{f}\PY{l+s+s2}{\PYZdq{}}\PY{l+s+s2}{Коефіцієнти ознак (area, bedrooms, bathrooms): }\PY{l+s+si}{\PYZob{}}\PY{n}{coefficients}\PY{l+s+si}{\PYZcb{}}\PY{l+s+s2}{\PYZdq{}}\PY{p}{)}
\end{Verbatim}
\end{tcolorbox}

\begin{Verbatim}[commandchars=\\\{\}]
	Вільний член (intercept): 4766729.236934193
	Коефіцієнти ознак (area, bedrooms, bathrooms): [3.78762791e+02 4.06820872e+05
	1.38604820e+06]
\end{Verbatim}

\begin{tcolorbox}[breakable, size=fbox, boxrule=1pt, pad at break*=1mm,colback=cellbackground, colframe=cellborder]
\prompt{In}{incolor}{11}{\boxspacing}
\begin{Verbatim}[commandchars=\\\{\}]
\PY{n}{plt}\PY{o}{.}\PY{n}{plot}\PY{p}{(}\PY{n}{history\PYZus{}J}\PY{p}{)}
\PY{n}{plt}\PY{o}{.}\PY{n}{xlabel}\PY{p}{(}\PY{l+s+s1}{\PYZsq{}}\PY{l+s+s1}{Номер ітерації}\PY{l+s+s1}{\PYZsq{}}\PY{p}{)}
\PY{n}{plt}\PY{o}{.}\PY{n}{ylabel}\PY{p}{(}\PY{l+s+s1}{\PYZsq{}}\PY{l+s+s1}{Значення функції втрат, J}\PY{l+s+s1}{\PYZsq{}}\PY{p}{)}
\PY{n}{plt}\PY{o}{.}\PY{n}{title}\PY{p}{(}\PY{l+s+s1}{\PYZsq{}}\PY{l+s+s1}{Графік зміни функції втрат від ітерації}\PY{l+s+s1}{\PYZsq{}}\PY{p}{)}
\PY{n}{plt}\PY{o}{.}\PY{n}{grid}\PY{p}{(}\PY{k+kc}{True}\PY{p}{)}

\PY{n}{plt}\PY{o}{.}\PY{n}{show}\PY{p}{(}\PY{p}{)}
\end{Verbatim}
\end{tcolorbox}

\begin{center}
	\adjustimage{max size={0.9\linewidth}{0.9\paperheight}}{hw3_files/hw3_22_0.png}
\end{center}
{ \hspace*{\fill} \\}

Аналітичний вираз для вектора вагових коефіцієнтів:

\[ \vec{w}^* = \left(\mathbf{X}^{\mathrm{T}} \mathbf{X}\right)^{-1}\mathbf{X}^{\mathrm{T}} \vec{y}. \]

Аналітичний метод надає точні значення коефіцієнтів, але для знаходження
вектора вагових коефіцієнтів за аналітичним методом потрібно обчислювати
обернену матрицю
\(\left(\mathbf{X}^{\mathrm{T}} \mathbf{X}\right)^{-1}\), що може
вимагати значних обчислювальних ресурсів. Зокрема, обчислення оберненої
матриці має складність порядку \(k^3\), де \(k\) - розмірність матриці.
Це може бути дуже витратним з обчислювальної точки зору.

Окрім того, якщо матриця \(\mathbf{X}^{\mathrm{T}} \mathbf{X}\) є погано
обумовленою, це означає, що власні числа цієї матриці близькі до нуля.
Погано обумовлена матриця може виникнути, наприклад, коли деякі ознаки
(стовпці матриці \(\mathbf{X}\)) мають високу кореляцію або
колінеарність. У таких випадках обчислення оберненої матриці може бути
непростим завданням, і воно може стати чисельно нестійким, що призводить
до неточностей і неправильних результатів.

Отже, в реальних задачах машинного навчання, де матриця
\(\mathbf{X}^{\mathrm{T}} \mathbf{X}\) може бути погано обумовленою або
великого розміру, аналітичний метод може бути невигідним через
обчислювальну складність та чисельну нестійкість, і частіше
використовуються інші методи оптимізації, такі як ітеративні методи
(наприклад, градієнтний спуск), які є більш ефективними та стійкими до
чисельних проблем.

\begin{tcolorbox}[breakable, size=fbox, boxrule=1pt, pad at break*=1mm,colback=cellbackground, colframe=cellborder]
\prompt{In}{incolor}{12}{\boxspacing}
\begin{Verbatim}[commandchars=\\\{\}]
\PY{n}{analitical\PYZus{}W} \PY{o}{=} \PY{n}{np}\PY{o}{.}\PY{n}{linalg}\PY{o}{.}\PY{n}{pinv}\PY{p}{(}\PY{n}{X}\PY{o}{.}\PY{n}{T} \PY{o}{@} \PY{n}{X}\PY{p}{)} \PY{o}{@} \PY{n}{X}\PY{o}{.}\PY{n}{T} \PY{o}{@} \PY{n}{Y}
\PY{n}{analitical\PYZus{}W}
\end{Verbatim}
\end{tcolorbox}

\begin{tcolorbox}[breakable, size=fbox, boxrule=.5pt, pad at break*=1mm, opacityfill=0]
	\prompt{Out}{outcolor}{12}{\boxspacing}
	\begin{Verbatim}[commandchars=\\\{\}]
		array([3.72448352e+02, 3.68974672e+05, 1.37031315e+06])
	\end{Verbatim}
\end{tcolorbox}

\section{\texorpdfstring{Алгоритми бібліотеки
	  \texttt{sklearn.linear\_model}}{Алгоритми бібліотеки sklearn.linear\_model}}\label{ux430ux43bux433ux43eux440ux438ux442ux43cux438-ux431ux456ux431ux43bux456ux43eux442ux435ux43aux438-sklearn.linear_model}

Алгоритми реалізують метод найменших квадратів (МНК)

\begin{tcolorbox}[breakable, size=fbox, boxrule=1pt, pad at break*=1mm,colback=cellbackground, colframe=cellborder]
\prompt{In}{incolor}{13}{\boxspacing}
\begin{Verbatim}[commandchars=\\\{\}]
\PY{n}{regressor} \PY{o}{=} \PY{n}{LinearRegression}\PY{p}{(}\PY{p}{)}\PY{o}{.}\PY{n}{fit}\PY{p}{(}\PY{n}{X}\PY{p}{,} \PY{n}{Y}\PY{p}{)}
\PY{n}{h\PYZus{}sk} \PY{o}{=} \PY{n}{regressor}\PY{o}{.}\PY{n}{predict}\PY{p}{(}\PY{n}{X}\PY{p}{)}
\end{Verbatim}
\end{tcolorbox}

\subsection{\texorpdfstring{Візуалізація за
		\texttt{sklearn.linear\_model}}{Візуалізація за sklearn.linear\_model}}\label{ux432ux456ux437ux443ux430ux43bux456ux437ux430ux446ux456ux44f-ux437ux430-sklearn.linear_model}

Цікаво побачити результати лінійної регресії.

\begin{tcolorbox}[breakable, size=fbox, boxrule=1pt, pad at break*=1mm,colback=cellbackground, colframe=cellborder]
\prompt{In}{incolor}{14}{\boxspacing}
\begin{Verbatim}[commandchars=\\\{\}]
\PY{c+c1}{\PYZsh{} Створимо маску для фільтрації даних з урахуванням фіксованих значень}
\PY{n}{f\PYZus{}1}\PY{p}{,} \PY{n}{f\PYZus{}2} \PY{o}{=} \PY{l+m+mi}{2}\PY{p}{,} \PY{l+m+mi}{1}

\PY{n}{mask} \PY{o}{=} \PY{p}{(}\PY{n}{X}\PY{p}{[}\PY{p}{:}\PY{p}{,} \PY{l+m+mi}{1}\PY{p}{]} \PY{o}{==} \PY{n}{f\PYZus{}1}\PY{p}{)} \PY{o}{\PYZam{}} \PY{p}{(}\PY{n}{X}\PY{p}{[}\PY{p}{:}\PY{p}{,} \PY{l+m+mi}{2}\PY{p}{]} \PY{o}{==} \PY{n}{f\PYZus{}2}\PY{p}{)}

\PY{c+c1}{\PYZsh{} Обираємо відповідні значення для фіксованих ознак і передбачені значення}
\PY{n}{selected\PYZus{}feature} \PY{o}{=} \PY{n}{X}\PY{p}{[}\PY{n}{mask}\PY{p}{]}\PY{p}{[}\PY{p}{:}\PY{p}{,} \PY{l+m+mi}{0}\PY{p}{]}
\PY{n}{h\PYZus{}sk\PYZus{}selected} \PY{o}{=} \PY{n}{h\PYZus{}sk}\PY{p}{[}\PY{n}{mask}\PY{p}{]}
\PY{n}{Y\PYZus{}selected} \PY{o}{=} \PY{n}{Y}\PY{p}{[}\PY{n}{mask}\PY{p}{]}

\PY{c+c1}{\PYZsh{} Гграфік залежності гіпотези від обраної ознаки за фіксованих значень інших ознак}
\PY{n}{plt}\PY{o}{.}\PY{n}{plot}\PY{p}{(}\PY{n}{selected\PYZus{}feature}\PY{p}{,} \PY{n}{h\PYZus{}sk\PYZus{}selected}\PY{p}{,} \PY{n}{label}\PY{o}{=}\PY{l+s+s2}{\PYZdq{}}\PY{l+s+s2}{Гипотеза}\PY{l+s+s2}{\PYZdq{}}\PY{p}{,} \PY{n}{color}\PY{o}{=}\PY{l+s+s1}{\PYZsq{}}\PY{l+s+s1}{red}\PY{l+s+s1}{\PYZsq{}}\PY{p}{)}
\PY{n}{plt}\PY{o}{.}\PY{n}{scatter}\PY{p}{(}\PY{n}{selected\PYZus{}feature}\PY{p}{,} \PY{n}{Y\PYZus{}selected}\PY{p}{,} \PY{n}{label}\PY{o}{=}\PY{l+s+s2}{\PYZdq{}}\PY{l+s+s2}{Дані}\PY{l+s+s2}{\PYZdq{}}\PY{p}{)}
\PY{n}{plt}\PY{o}{.}\PY{n}{xlabel}\PY{p}{(}\PY{l+s+s2}{\PYZdq{}}\PY{l+s+s2}{Обрана ознака (area)}\PY{l+s+s2}{\PYZdq{}}\PY{p}{)}
\PY{n}{plt}\PY{o}{.}\PY{n}{ylabel}\PY{p}{(}\PY{l+s+s2}{\PYZdq{}}\PY{l+s+s2}{Передбачені значення}\PY{l+s+s2}{\PYZdq{}}\PY{p}{)}
\PY{n}{plt}\PY{o}{.}\PY{n}{title}\PY{p}{(}\PY{l+s+sa}{f}\PY{l+s+s2}{\PYZdq{}}\PY{l+s+s2}{Графік залежності гіпотези від площі (area) (за фіксованих bedrooms =}\PY{l+s+si}{\PYZob{}}\PY{n}{f\PYZus{}1}\PY{l+s+si}{\PYZcb{}}\PY{l+s+s2}{ і bathrooms=}\PY{l+s+si}{\PYZob{}}\PY{n}{f\PYZus{}2}\PY{l+s+si}{\PYZcb{}}\PY{l+s+s2}{)}\PY{l+s+s2}{\PYZdq{}}\PY{p}{)}
\PY{n}{plt}\PY{o}{.}\PY{n}{legend}\PY{p}{(}\PY{p}{)}
\PY{n}{plt}\PY{o}{.}\PY{n}{grid}\PY{p}{(}\PY{k+kc}{True}\PY{p}{)}
\PY{n}{plt}\PY{o}{.}\PY{n}{show}\PY{p}{(}\PY{p}{)}
\end{Verbatim}
\end{tcolorbox}

\begin{center}
	\adjustimage{max size={0.9\linewidth}{0.9\paperheight}}{hw3_files/hw3_28_0.png}
\end{center}
{ \hspace*{\fill} \\}



\subsection{Результати}\label{ux440ux435ux437ux443ux43bux44cux442ux430ux442ux438}

\begin{enumerate}
	\def\labelenumi{\arabic{enumi}.}
	\tightlist
	\item
	      За ``self-made'' алгоритмом градієнтного спуску
\end{enumerate}

\begin{tcolorbox}[breakable, size=fbox, boxrule=1pt, pad at break*=1mm,colback=cellbackground, colframe=cellborder]
\prompt{In}{incolor}{15}{\boxspacing}
\begin{Verbatim}[commandchars=\\\{\}]
\PY{n+nb}{print}\PY{p}{(}\PY{l+s+sa}{f}\PY{l+s+s2}{\PYZdq{}}\PY{l+s+s2}{Коефіцієнти ознак (area, bedrooms, bathrooms): }\PY{l+s+si}{\PYZob{}}\PY{n}{coefficients}\PY{l+s+si}{\PYZcb{}}\PY{l+s+s2}{\PYZdq{}}\PY{p}{)}
\end{Verbatim}
\end{tcolorbox}

\begin{Verbatim}[commandchars=\\\{\}]
Коефіцієнти ознак (area, bedrooms, bathrooms): [3.78762791e+02 4.06820872e+05
1.38604820e+06]
\end{Verbatim}

\begin{enumerate}
	\def\labelenumi{\arabic{enumi}.}
	\setcounter{enumi}{1}
	\tightlist
	\item
	      За аналітичним розрахунком
\end{enumerate}

\begin{tcolorbox}[breakable, size=fbox, boxrule=1pt, pad at break*=1mm,colback=cellbackground, colframe=cellborder]
\prompt{In}{incolor}{16}{\boxspacing}
\begin{Verbatim}[commandchars=\\\{\}]
\PY{n+nb}{print}\PY{p}{(}\PY{l+s+sa}{f}\PY{l+s+s2}{\PYZdq{}}\PY{l+s+s2}{Коефіцієнти ознак (area, bedrooms, bathrooms): }\PY{l+s+si}{\PYZob{}}\PY{n}{analitical\PYZus{}W}\PY{l+s+si}{\PYZcb{}}\PY{l+s+s2}{\PYZdq{}}\PY{p}{)}
\end{Verbatim}
\end{tcolorbox}

\begin{Verbatim}[commandchars=\\\{\}]
	Коефіцієнти ознак (area, bedrooms, bathrooms): [3.72448352e+02 3.68974672e+05
	1.37031315e+06]
\end{Verbatim}

\begin{enumerate}
	\def\labelenumi{\arabic{enumi}.}
	\setcounter{enumi}{2}
	\tightlist
	\item
	      За МНК із бібліотеки \texttt{scisklearn.linear\_model}
\end{enumerate}

\begin{tcolorbox}[breakable, size=fbox, boxrule=1pt, pad at break*=1mm,colback=cellbackground, colframe=cellborder]
\prompt{In}{incolor}{17}{\boxspacing}
\begin{Verbatim}[commandchars=\\\{\}]
\PY{n+nb}{print}\PY{p}{(}\PY{l+s+sa}{f}\PY{l+s+s2}{\PYZdq{}}\PY{l+s+s2}{Коефіцієнти ознак (area, bedrooms, bathrooms): }\PY{l+s+si}{\PYZob{}}\PY{n}{regressor}\PY{o}{.}\PY{n}{coef\PYZus{}}\PY{l+s+si}{\PYZcb{}}\PY{l+s+s2}{\PYZdq{}}\PY{p}{)}
\end{Verbatim}
\end{tcolorbox}

\begin{Verbatim}[commandchars=\\\{\}]
Коефіцієнти ознак (area, bedrooms, bathrooms): [3.78762754e+02 4.06820034e+05
1.38604950e+06]
\end{Verbatim}

\subsection{Вартість
	квартири}\label{ux432ux430ux440ux442ux456ux441ux442ux44c-ux43aux432ux430ux440ux442ux438ux440ux438}

Розглянемо конкретний випадок. Зробимо передбачення ціни на квартиру яка
має характеристики \(x_1 = 7420\), \(x_2 = 3\), \(x_3 = 1\).

\begin{tcolorbox}[breakable, size=fbox, boxrule=1pt, pad at break*=1mm,colback=cellbackground, colframe=cellborder]
\prompt{In}{incolor}{18}{\boxspacing}
\begin{Verbatim}[commandchars=\\\{\}]
\PY{n}{my\PYZus{}X} \PY{o}{=} \PY{n}{np}\PY{o}{.}\PY{n}{array}\PY{p}{(}\PY{p}{[}\PY{p}{[}\PY{l+m+mi}{7420}\PY{p}{,} \PY{l+m+mi}{3}\PY{p}{,} \PY{l+m+mi}{1}\PY{p}{]}\PY{p}{]}\PY{p}{)}
\end{Verbatim}
\end{tcolorbox}

\begin{enumerate}
	\def\labelenumi{\arabic{enumi}.}
	\tightlist
	\item
	      Наша функція гіпотези
\end{enumerate}

\begin{tcolorbox}[breakable, size=fbox, boxrule=1pt, pad at break*=1mm,colback=cellbackground, colframe=cellborder]
\prompt{In}{incolor}{19}{\boxspacing}
\begin{Verbatim}[commandchars=\\\{\}]
\PY{n+nb}{print}\PY{p}{(}\PY{l+s+sa}{f}\PY{l+s+s2}{\PYZdq{}}\PY{l+s+s2}{Ціна за квартиру }\PY{l+s+si}{\PYZob{}}\PY{n}{h}\PY{p}{(}\PY{n}{coefficients}\PY{p}{,}\PY{+w}{ }\PY{n}{my\PYZus{}X}\PY{p}{)}\PY{p}{[}\PY{l+m+mi}{0}\PY{p}{]}\PY{l+s+si}{:}\PY{l+s+s2}{.0f}\PY{l+s+si}{\PYZcb{}}\PY{l+s+s2}{\PYZdq{}}\PY{p}{)}
\end{Verbatim}
\end{tcolorbox}

\begin{Verbatim}[commandchars=\\\{\}]
	Ціна за квартиру 5416931
\end{Verbatim}

\begin{enumerate}
	\def\labelenumi{\arabic{enumi}.}
	\setcounter{enumi}{1}
	\tightlist
	\item
	      Функція бібліотеки \texttt{scisklearn.linear\_model}
\end{enumerate}

\begin{tcolorbox}[breakable, size=fbox, boxrule=1pt, pad at break*=1mm,colback=cellbackground, colframe=cellborder]
\prompt{In}{incolor}{20}{\boxspacing}
\begin{Verbatim}[commandchars=\\\{\}]
\PY{n+nb}{print}\PY{p}{(}\PY{l+s+sa}{f}\PY{l+s+s2}{\PYZdq{}}\PY{l+s+s2}{Ціна за квартиру }\PY{l+s+si}{\PYZob{}}\PY{n}{regressor}\PY{o}{.}\PY{n}{predict}\PY{p}{(}\PY{n}{my\PYZus{}X}\PY{p}{)}\PY{p}{[}\PY{l+m+mi}{0}\PY{p}{]}\PY{l+s+si}{:}\PY{l+s+s2}{.0f}\PY{l+s+si}{\PYZcb{}}\PY{l+s+s2}{\PYZdq{}}\PY{p}{)}
\end{Verbatim}
\end{tcolorbox}

\begin{Verbatim}[commandchars=\\\{\}]
	Ціна за квартиру 5243758
\end{Verbatim}

\subsection{Висновки}\label{ux432ux438ux441ux43dux43eux432ux43aux438}

Відмінності в значеннях коефіцієнтів між методами (градієнтним спуском,
аналітичним методом і МНК) може бути зумовлена відмінностями в підходах
і параметрах кожного методу.

У цьому контексті:

\begin{enumerate}
	\def\labelenumi{\arabic{enumi}.}
	\item
	      Градієнтний спуск - ітеративний метод, який залежить від початкової
	      ініціалізації та параметрів навчання, таких як швидкість навчання.
	      Результати можуть варіюватися залежно від цих факторів.
	\item
	      Аналітичний метод - знаходить точне аналітичне рішення і не залежить
	      від параметрів навчання.
	\item
	      Метод найменших квадратів (МНК) - також знаходить точне рішення і не
	      вимагає налаштування параметрів навчання.
\end{enumerate}

Відмінності у вагових коефіцієнтах можуть бути спричинені як
відмінностями в методах оптимізації, так і в особливостях даних, таких
як наявність викидів, шумів або кореляцій між ознаками. Однак важливо
зазначити, що за правильного налаштування й обробки даних відмінності в
коефіцієнтах між цими методами мають бути незначними, і всі три методи
мають давати схожі результати в контексті лінійної регресії.




% Add a bibliography block to the postdoc



\end{document}
